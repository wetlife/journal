\section*{2015.08.21}
\subsection*{#loveLife}
This is some stuff. The stuff is words, really. The amazing thing is I get to string them together however I please.

I am so in love because of who Amanda Chadderdon is and how lovingly Amanda has behaved toward me. I am scared to tell her this because I think it may scare her off. I don't know what to do.

I say "I'm not sure what to do" as though, and I want to joke that "I \textbf{AM} sure I'll be happy however my life pans out." but I say "I'm not sure" as though...I LOST THE THOUGHT OVER THE COURSE OF HOURS...

\subsection*{#lifePlan}
I wanna journal about what I'll do with my life. I have come to the conclusion that...CONTINUED ON NEXT

\section*{2015.08.21at06:57}
\subsection*{#lifePlan} CONTINUED FROM YESTERDAY'S THOUGHT:
"I have come to the conclusion that..." I should create a vegan cheese while I'm looking for work and working off student loans as a cab-driver. In order to plan and verify feasability, I calculate some basic time requirements imposed by all the things I want to do(after this table, a current plan is to be kept in tabulated form in weeklyTimeTable.tex):

\begin{table}
\caption{\label{tab:originalWeeklyAndDailyTimeTable}Weekly and Daily Time-Table}
\begin{tabular}{|r|r|l|}
\hline Hours/Week&Hours/Day&Activity\\\hline
7&1&hygiene\\
7&1&workout\\
49&12$\times$4+1&drive a cab\\
14&2&social and play time\\
42&6&sleep\\
21&3&study graduate physics courses\\
28&4&journal and invent\\\hline
\end{tabular}\end{table}

This gives $7 \text{days} \times 24 \text{hours} = 168$ hours per week, accounting for the all hours in a week. I intend to investigate using food, sleep, and drugs to augment my productivity in learning and inventing. Here, inventing can be broadly interpretted.

\subsection*{#invention}
Invention could mean creating a vegan cheese by chemical synthesizing a substrate with the desired percentages and types of fat, protein, and carbohydrate. Invention could mean designing a belt for runners to store an empty or full water bladder in, as well as shopping bags, a phone, pens, keys, clean and dry clothes, and other items. Invention could mean designing a pair of gloves with all the things listed for the belt attached as well as carrying pouches to provide runners with an ergonomic upper-body workout while running.

\section*{2015.08.26at15:18}
\subsection*{#invention, #softwareInScience, #hackingForScience}
Make idealab: a database of possibly networked ideas, unsolved problems, and solutions which are easily searched by users. An idea of a use-case: lab A at some institution has a problem which impedes its progress and so it checks with idealab for a solution. Because the solution to this probelem hasn't been catalogued in an appropriately findable manner, lab A submits this problem to idealab.

Lab B has problem P. Lab B searches for a solution using standard methods. lab B finds lab A's record of problem P on ideaLab. Lab B registers another incidence of problem P.

IdeaLab has a system...

\section*{2015.08.27at22:53}
\subsection*{#invention, #softwareInScience, #hackingForScience, #ideaLab}
{/bf Continued from 2015.08.26at15:18}
IdeaLab has a system to rank ideas relative to one another. An idea's rank is a function that increases with the number of users who had this idea. Rank increases with the number of viewers it has.

Think of additional variables that rank varies with. The single most important task in creating ideaLab is to make idea browsing and discovery of ideas work well. IdeaLab is a great way to bring ideas of how to help the human endeavor to the layperson.

Suppose Bob the bricklayer has an idea which helps Portland Cement make a better cement. If only there was a framework for the sharing of ideas to be witnessed and paid for as a percentage of calculated increased profit. Players are motivated to pay handsomely for ideas to encourage new ideas. This an alternative to a conventional research and development department,

IdeaLab's job in this is:  provide the easy access to the hosted ideas, problems, and solutions.[I wanted to add more here about idealab and I wanna work on the idealab site as a web-app. Make this as a ruby, php, or python site.]

\section*{2015.08.31at14:35}
\subsection*{#journal, #DIY-ence, #ideaLab}
People can run independent studies and experiments by throwing food to a group of ducks and recording the responses of the groups. When observing a small group, it seems to observe whether there is altruism displayed. For example, when a large clump of bread is thrown to the group, does the group let all members get some part of the clump or do the dominant ducks hoard the resource?

I want to read "open access" journals. This will keep me up on subject matters' current evolution.

\section*{2015.09.15at20:15}
\subsection*{#journal}
I find myself wondering why are kthese prople the way they are. This is about the most ridiculous thing Ive seen. I have finally made myself come out and write and I actually am thinking "this is all I can come up with."

Language is a beautiful "arbitrary," as in "it was decided," and "mysterious," as in "what things can a study of this thing[, language,] tell us about nature. It beautiful that a finite set of symbols and rules can describe such a seemingly infinite set of things. This is a beauty worth exploring.

\subsection*{lyrics}
She's so sweet but I'm just me.
She never told me what to be.
If I am wrong then that's me and that's all that we'll be, but I love her.
Then I'm gone and that's all that I'll see of her.

What's there to love when she's gone?
I can't say what's not there.
I'm here.
I'll love me.
I love me.
I love to be
everything
never mean
know what I mean?

\section*{2015.10.09at10:15}
\subsection*{ #schedule #ideaConsulting }
I have decided to fulfill my dreams with the time scheduled for sleeping. This means I will read interesting and promising books with this time. I will write my thoughts and video record journals when necessary because typing can't keep up with the rate of my thoughts.

On that note, I should start playing with a typing tutor. I want to find a good one. Back to a thought from the previous paragraph, Google could make software for video journaling.

When thinking about the idea of an unaffiliated, freelancing individual having ideas that benefit a company, I think I should pioneer an industry I dub idea consulting.

\section*{ 2016.01.05at14:28 }
\subsection*{ #inDefenseOfPeace }

I have not seen a disagreement that I thought was worth lives. Does fighting make a person tough? Forfeiting thinking for fighting appears to turn righteousness into wrongfulness. Can we think of a better way?

What if we use the money, energy, and passion currently dedicated to an ironically named Department of Defense to find appropriate resolutions to problems? What are you willing to spend to find resolutions which don't cost lives to raise hate?

Peaceful resolutions may be hard to find. People commend soldiers for participating in fights we all fund. What if people spent the same money and lives on clever diplomacy that avoids wars? What if we use love to spread love?

One idea is to reward diplomacy. What if there was a system to which any person can contribute diplomatic ideas and be rewarded for bringing a good idea? What if people could work together to find an elusive peaceful resolution?

Wikipedia's Software Industry article reads "according to industry analyst Gartner, the size of the worldwide software industry in 2013 was US\$407.3 billion." Even in this well funded industry, many of the most effective tools are created and actively maintained by thousands of volunteers. I am talking about open-sourced software.

The term opensource refers to software whose source-code is freely editable. This allows any person to contribute a resolution to any issue. In this way, the opensource community offers competitive solutions which are licensed to remain freely editable by posterity.

Can opensource show us how to collaborate on ideas for peaceful resolutions? Opensource provides solutions which competed with with the leading products of a US\$400 billion industry with notable successes. Opensource software dominates the market for web servers and secure cheap personal computing playgrounds.

\section*{ 2016.02.02at22:13 }
\subsection*{ #mathematicalBreadcrumbs }
I hope that each who is in wonder at the majesties of either math or science one day realizes the beauty that complex exponentials store the arbitrary but beatifully succinct connection between algebra and rotations.

\section*{ 2016.02.06at18:34 }
\subsection*{ financialPlannery investmentStrategy }
Plan to save money at the rate $R$ dollars per time period(e.g. year, month, week, day,..). Put money away in savings in quantity $S$. Purchase in quantity $P$ for repairing the house with top quality fixes that increase equity or other acts which increase assets.

The killer idea here is to count asset-increasing-work as money saved in case saving everything over a reasonable spending limit doesn't work. This is, however, the preferred saving plan: 
\begin{itemized}
\item spend within spending limit $L$;
\item let $E$ be the amount earned during the period;
\item let $J$ be the current project, E.G. 
\item $E = $
\end{itemized}

With these definitions let $R=S+P$. 
