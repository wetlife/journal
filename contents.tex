\section*{2015.08.21}
\subsection*{#loveLife}
This is some stuff. The stuff is words, really. The amazing thing is I get to string them together however I please.

I am so in love because of who Amanda Chadderdon is and how lovingly Amanda has behaved toward me. I am scared to tell her this because I think it may scare her off. I don't know what to do.

I say "I'm not sure what to do" as though, and I want to joke that "I \textbf{AM} sure I'll be happy however my life pans out." but I say "I'm not sure" as though...I LOST THE THOUGHT OVER THE COURSE OF HOURS...

\subsection*{#lifePlan}
I wanna journal about what I'll do with my life. I have come to the conclusion that...CONTINUED ON NEXT

\section*{2015.08.21at06:57}
\subsection*{#lifePlan} CONTINUED FROM YESTERDAY'S THOUGHT:
"I have come to the conclusion that..." I should create a vegan cheese while I'm looking for work and working off student loans as a cab-driver. In order to plan and verify feasability, I calculate some basic time requirements imposed by all the things I want to do(after this table, a current plan is to be kept in tabulated form in weeklyTimeTable.tex):

\begin{table}
\caption{\label{tab:originalWeeklyAndDailyTimeTable}Weekly and Daily Time-Table}
\begin{tabular}{|r|r|l|}
\hline Hours/Week&Hours/Day&Activity\\\hline
7&1&hygiene\\
7&1&workout\\
49&12$\times$4+1&drive a cab\\
14&2&social and play time\\
42&6&sleep\\
21&3&study graduate physics courses\\
28&4&journal and invent\\\hline
\end{tabular}\end{table}

This gives $7 \text{days} \times 24 \text{hours} = 168$ hours per week, accounting for the all hours in a week. I intend to investigate using food, sleep, and drugs to augment my productivity in learning and inventing. Here, inventing can be broadly interpretted.

\subsection*{#invention}
Invention could mean creating a vegan cheese by chemical synthesizing a substrate with the desired percentages and types of fat, protein, and carbohydrate. Invention could mean designing a belt for runners to store an empty or full water bladder in, as well as shopping bags, a phone, pens, keys, clean and dry clothes, and other items. Invention could mean designing a pair of gloves with all the things listed for the belt attached as well as carrying pouches to provide runners with an ergonomic upper-body workout while running.

\section*{2015.08.26at15:18}
\subsection*{#invention, #softwareInScience, #hackingForScience}
Make idealab: a database of possibly networked ideas, unsolved problems, and solutions which are easily searched by users. An idea of a use-case: lab A at some institution has a problem which impedes its progress and so it checks with idealab for a solution. Because the solution to this probelem hasn't been catalogued in an appropriately findable manner, lab A submits this problem to idealab.

Lab B has problem P. Lab B searches for a solution using standard methods. lab B finds lab A's record of problem P on ideaLab. Lab B registers another incidence of problem P.

IdeaLab has a system...

\section*{2015.08.27at22:53}
\subsection{#invention, #softwareInScience, #hackingForScience, #ideaLab}
{/bf Continued from 2015.08.26at15:18}
IdeaLab has a system to rank ideas relative to one another. An idea's rank is a function that increases with the number of users who had this idea. Rank increases with the number of viewers it has.

Think of additional variables that rank varies with. The single most important task in creating ideaLab is to make idea browsing and discovery of ideas work well. IdeaLab is a great way to bring ideas of how to help the human endeavor to the layperson.

Suppose Bob the bricklayer has an idea which helps Portland Cement make a better cement. If only there was a framework for the sharing of ideas to be witnessed and paid for as a percentage of calculated increased profit. Players are motivated to pay handsomely for ideas to encourage new ideas. This an alternative to a conventional research and development department,

IdeaLab's job in this is:  provide the easy access to the hosted ideas, problems, and solutions.[I wanted to add more here about idealab and I wanna work on the idealab site as a web-app. Make this as a ruby, php, or python site.]

\section*{2015.08.31at14:35}
\subsection{#Journal, #DIY-ence, #ideaLab}
People can run independent studies and experiments by throwing food to a group of ducks and recording the responses of the groups. When observing a small group, it seems to observe whether there is altruism displayed. For example, when a large clump of bread is thrown to the group, does the group let all members get some part of the clump or do the dominant ducks hoard the resource?

I want to read "open access" journals. This will keep me up on subject matters current evolution.
