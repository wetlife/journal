\section*{Journal Stream Started 2015.08.21}

\subsection*{\#loveLife}
This is some stuff. The stuff is words, really. The amazing thing is I get to string them together however I please.

I am so in love because of who Amanda Chadderdon is and how lovingly Amanda has behaved toward me. I am scared to tell her this because I think it may scare her off. I don't know what to do.

I say "I'm not sure what to do" as though, and I want to joke that "I \textbf{AM} sure I'll be happy however my life pans out." but I say "I'm not sure" as though...I LOST THE THOUGHT OVER THE COURSE OF HOURS...

\subsection*{\#lifePlan}
I wanna journal about what I'll do with my life. I have come to the conclusion that...CONTINUED ON NEXT

\section*{2015.08.21at06:57}
\subsection*{\#lifePlan} CONTINUED FROM YESTERDAY'S THOUGHT:
"I have come to the conclusion that..." I should create a vegan cheese while I'm looking for work and working off student loans as a cab-driver. In order to plan and verify feasability, I calculate some basic time requirements imposed by all the things I want to do(after this table, a current plan is to be kept in tabulated form in weeklyTimeTable.tex):

\begin{table}
\caption{\label{tab:originalWeeklyAndDailyTimeTable}Weekly and Daily Time-Table}
\begin{tabular}{|r|r|l|}
\hline
Hours/Week&Hours/Day&Activity\\
\hline
7&1&hygiene\\
7&1&workout\\
49&12$\times$4+1&drive a cab\\
14&2&social and play time\\
42&6&sleep\\
21&3&study graduate physics courses\\
28&4&journal and invent\\
\hline
\end{tabular}\end{table}

This gives $7 \text{days} \times 24 \text{hours} = 168$ hours per week, accounting for the all hours in a week. I intend to investigate using food, sleep, and drugs to augment my productivity in learning and inventing. Here, inventing can be broadly interpretted.

\subsection*{\#invention}
Invention could mean creating a vegan cheese by chemical synthesizing a substrate with the desired percentages and types of fat, protein, and carbohydrate. Invention could mean designing a belt for runners to store an empty or full water bladder in, as well as shopping bags, a phone, pens, keys, clean and dry clothes, and other items. Invention could mean designing a pair of gloves with all the things listed for the belt attached as well as carrying pouches to provide runners with an ergonomic upper-body workout while running.

\section*{2015.08.26at15:18}
\subsection*{\#invention, \#softwareInScience, \#hackingForScience}
Make idealab: a database of possibly networked ideas, unsolved problems, and solutions which are easily searched by users. An idea of a use-case: lab A at some institution has a problem which impedes its progress and so it checks with idealab for a solution. Because the solution to this probelem hasn't been catalogued in an appropriately findable manner, lab A submits this problem to idealab.

Lab B has problem P. Lab B searches for a solution using standard methods. lab B finds lab A's record of problem P on ideaLab. Lab B registers another incidence of problem P.

IdeaLab has a system...

\section*{2015.08.27at22:53}
\subsection*{\#invention, \#softwareInScience, \#hackingForScience, \#ideaLab}
{\bf Continued from 2015.08.26at15:18}
IdeaLab has a system to rank ideas relative to one another. An idea's rank is a function that increases with the number of users who had this idea. Rank increases with the number of viewers it has.

Think of additional variables that rank varies with. The single most important task in creating ideaLab is to make idea browsing and discovery of ideas work well. IdeaLab is a great way to bring ideas of how to help the human endeavor to the layperson.

???LINES MISSING
???LINES MISSING
???LINES MISSING
???LINES MISSING
???LINES MISSING
???LINES MISSING
???LINES MISSING
???LINES MISSING
???LINES MISSING
???LINES MISSING
???LINES MISSING
???LINES MISSING
???LINES MISSING
???LINES MISSING
???LINES MISSING
???LINES MISSING
???LINES MISSING
???LINES MISSING
???LINES MISSING
???LINES MISSING
???LINES MISSING

\section*{ 2016.10.05at05:21 }
\subsection*{ #writtenWithBlahScript }
Investigate genetic modification's possible role in the seemingly increased incidence of cancer.

\section*{ 2016.10.05at05:26 }
\subsection*{ #writtenWithBlahScript }
I want to investigate how to escape quotes when using blah to journal one liners from the command-line.

\section*{ 2016.10.05at05:44 }
\subsection*{ #writtenWithBlahScript }
Deliver the message that whenever difficulty with communication arises one may recall what is trying to be said and resort to the "true sentence-prototype" of <subject> <verb> <object>.

\section*{ 2016.10.05at05:49 }
\subsection*{ #writtenWithBlahScript }
text for journal
\nThe above command appends a time-stamp and text for journal into my journal.

\section*{ 2016.10.05at06:55 }
\subsection*{ #writtenWithBlahScript }
I want to discuss my wildest dreams with Professor David Roundy, Professor Tevian Dray, Professor Corinne Manogue, Professor Weihong Qiu, Professor Oksana Ostroverkhova iCan'tSpellHerLastName, Professor Matt Graham, Professor Ethan Minot, Professor Yun Shik Lee, Instructor Chris Coffin, Instructor KC Walsh, the CS professor that let me borrow the arduino board when I sat in on his course on assembly, Instructor Jim Ketter, Professor Janet Tate, Professor Henry Jansen, Instructor Sean Hetcherson, Professor Holly Swisher, Professor Ren Guo, Professor Ralph E. Showalter, Professor Malgorzata Peszynska, Professor Clayton Petche, Instructor Scott L. Peterson, Professor Radu Dascaliuc, Instructor Fred Brick, and others both in person and via email.
>>>>>>> refs/remotes/origin/master
\section*{ 2016.12.14at22:14 }
\subsection*{ #writtenWithBlahScript }


\section*{ 2016.12.15at18:30 }
\subsection*{ #writtenWithBlahScript }
there are things that never get said because not interesting like a new concept.there

\section*{ 2016.12.16at22:11 }
\subsection*{ = '~/journal/ }
This title was a mistake when learning a new prompt named xonsh. It feels like it should in principle be possible to design the currently best ever computer interface based on the keyboard.

I of course say this because I've already done it, though it seems like it would take conscious thought to undertake. I need to coach myself to victory here because my reach into wonderful possibilities seems extended by wealth.

The interface is simply a set of keyboard bindings close to those of vi. There is no copyrighted interaction here. Download costs pay custodial engineers-in-training. The teachers will guide the educations of these interns.

It is like a programming language but needs no infrastructure. The people's desire to use such a wonderful interface will carry this mode of interaction with the computer to wherever they go. It needs something like a new language of documentation to maintain portability.

The documentation will be just the needed words to communicate the necessary functions completely. Discussions of essentials and really anything belong outside the documentation. I provide an example used to describe the current vission of this interface.



\section*{ 2016.12.21at20:58 }
\subsection*{ A trip for the generations. }
This is me talking through my computer. I definitely will be better off if I idea-mine these journals. Idea-mine as a mine of these ideas of mine.

\section*{ 2017.01.08at02:33 }
\subsection*{ I need to write a book called 4ever1 }
I want to make the outline for the science, math, logic, analysis, and definitions section. This book should explain to a child all of the language, logic, math, physics, chemistry, fluid mechanics, and whatever other things I understand better at a later time. here iWrite

PS thank you Kore, dated now, 2017.01.08at02:33

\section*{ 2017.01.16at12:39 }
\subsection*{ A Machine-learning Approach to the Management of Time }
Split the day into three sectors of eight hours, as I have heard is customarily done in Japan. This gives eight hours for sleep, eight hours for work, and eight hours of other time. This is a way to divide one's daily time into three things many find to be necessary in his or her life.

I can sleep eight hours, work eight hours, excercise two hours, take personal care for one hour, allow three hours of incidental time such as maintaining my home or tending to bills. After these allowances I still have two hours of time everyday to use for arbitrary activities.

I want to journal and learn every day. These are what I find fulfilling in the course of a day. These things are fulfilling when I can manage the rest of my life satisfactorily.

I want to experiment sticking to this regimen for two weeks. I want to review my objective productivity in each sector by some measurable quantity, I want to review my performance at work, review my amount of energy, review my health while remembering that nourishment plays a huge role in health and must be sufficient, review my learning over this span of time, and review my performance on utilizing the time for each task as intended.

What I am doing is setting up a system to analytically address my fulfillment. This is the end goal in life: to live in a fulfilling manner. I define fulfillment as satisfaction with one's time. Satisfaction can be the performance measure gained from the reviews.

View the task as living well. I define living well by maintaining health, tending to the future, and feeling fulfilled. Machine learning can be utilized to maximize the fruits of one's labors when defining things this way. What one needs is a starting point to work from. A starting point has been provided above. 

Consider Tom M. Mitchell's definition of machine-learning: "A computer program is said to learn from experience E with respect to some class of tasks T and performance measure P if its performance at tasks in T, as measured by P, improves with experience E." Let T be living a day well. Let P be reviews. Let E be days lived. This defines living well as a problem of a learning machine.

Humans can function as machines learning to live better. Humans already get to define how they each measure living well. This framework allows a quantitative means for this task. This quantitative means is used in a feedback-loop for continuous improvement at task T: living life fulfillingly.

\section*{ 2017.01.27at23:16 }
\subsection*{ Great minds throughout time have been the first to think of things. }
All the wonderful ever first seen have been seen already. This means the target of seeking truth moves. This means the seeking of new truths never before entered into the historical record moves further and down a sometimes counterintuitively branched logical rabbit hole of the state of predictions.

\section*{ 2017.01.29at11:52 }
\subsection*{ I love science because... }
I love science because all of the wonderful things it does. What is being said with the word "because" in the previous sentence? One interpretation could be that the sentence just makes one feel good.

On analysis, however, we get to the good stuff. That is the fun and other utility are maximized. I want to be reading about python's natural language processing module. I have this belief that if I love Jenna then I should say something now because what if she marries Alex? I should lay everything on her as the hero of my own story could be heroically romanticized to do. I should say only the friendliest, warmest things because that is what the hero of my story will do.

I simply want to get Jenna's excitement about topics scientific. This want is because I've experienced how energizing and pleasing it is to be adored by Jenna. 

\section*{ 2017.02.24at18:59 }
\subsection*{ Make time to read me. }

\section*{ 2017.02.24at21:35 }
\subsection*{ workHere }
Cluster descriptions of Sustainable and Healthy Community's(SHC's)tools in a space $\mathbb{S}=$ roles $\cross$ subroles $\cross$ fundamental objectives $\cross$ objectives $\cross$ concepts. Vary the number of clusters used in a k-means algorithm within $\mathbb{S}$. Give clustering data and code to the developers of Decision Path Library(DPL). This tool can be a means of crowd-sourcing the development of DPL by informing the DPL's labeling of points in $\mathbb{S}$ with concepts.

DPL can utilize the feedback users give that additional labels are desired. The SHC-Tools-Inventory webapp can use the DPL's labels and allow users to seve tools and dismiss tools from the cart. Tools that strongly correlate weak cart-scores with some subspace $\mathbb{S}'$ of $\mathbb{S}$ tell DPL not to specify $l$ in $\mathbb{S}'$. Weak cart-scores indicate lack of accuracy in mapping concepts to tools. 

A weak cart-score is a small ratio of cart-additions to total suggestions of the label.

\section*{ 2017.03.30at21:34 }
\subsection*{ Make a million dollars right now. }
I propose this for as soon as I get a chance: begin my fortune by going to Ryan, Sutie's grandson-in-law who works at Google, with the idea for Google to make a collaborative compositional tool used to create beautiful openly sourced music and collaborative educational masterworks of mastery from many masters' remasterings.

\section*{ 2017.03.31at23:53 }
\subsection*{ Jenna }
Jenna, I've been thinking about you. I've decided a thousand times what to do. I have resolved to come down to California, where I saw you were going to school on your Facebook profile. I want to wish Alex the best because he is the man you've chosen. Complete is the only respect I will show your opinion.

\section*{ 2017.04.01at01:49 }
\subsection*{ millionDollarIdea }
Go to Microsoft as a partner in making an interface that's accessible from this beautiful keyboard thing. Go to Google about reimagining Google Play as an instrument on the web; that is an instrument through which anyone connected to the web can compose music, respond to compositions, AA
