section*{Journal Stream Started 2015.08.21}

\subsection*{\#loveLife}
This is some stuff. The stuff is words, really. The amazing thing is I get to string them together however I please.

I am so in love because of who Amanda Chadderdon is and how lovingly Amanda has behaved toward me. I am scared to tell her this because I think it may scare her off. I don't know what to do.

I say "I'm not sure what to do" as though, and I want to joke that "I \textbf{AM} sure I'll be happy however my life pans out." but I say "I'm not sure" as though...I LOST THE THOUGHT OVER THE COURSE OF HOURS...

\subsection*{\#lifePlan}
I wanna journal about what I'll do with my life. I have come to the conclusion that...CONTINUED ON NEXT

\section*{2015.08.21at06:57}
\subsection*{\#lifePlan} CONTINUED FROM YESTERDAY'S THOUGHT:
"I have come to the conclusion that..." I should create a vegan cheese while I'm looking for work and working off student loans as a cab-driver. In order to plan and verify feasability, I calculate some basic time requirements imposed by all the things I want to do(after this table, a current plan is to be kept in tabulated form in weeklyTimeTable.tex):

\begin{table}
\caption{\label{tab:originalWeeklyAndDailyTimeTable}Weekly and Daily Time-Table}
\begin{tabular}{|r|r|l|}
\hline
Hours/Week&Hours/Day&Activity\\
\hline
7&1&hygiene\\
7&1&workout\\
49&12$\times$4+1&drive a cab\\
14&2&social and play time\\
42&6&sleep\\
21&3&study graduate physics courses\\
28&4&journal and invent\\
\hline
\end{tabular}\end{table}

This gives $7 \text{days} \times 24 \text{hours} = 168$ hours per week, accounting for the all hours in a week. I intend to investigate using food, sleep, and drugs to augment my productivity in learning and inventing. Here, inventing can be broadly interpretted.

\subsection*{\#invention}
Invention could mean creating a vegan cheese by chemical synthesizing a substrate with the desired percentages and types of fat, protein, and carbohydrate. Invention could mean designing a belt for runners to store an empty or full water bladder in, as well as shopping bags, a phone, pens, keys, clean and dry clothes, and other items. Invention could mean designing a pair of gloves with all the things listed for the belt attached as well as carrying pouches to provide runners with an ergonomic upper-body workout while running.

\section*{2015.08.26at15:18}
\subsection*{\#invention, \#softwareInScience, \#hackingForScience}
Make idealab: a database of possibly networked ideas, unsolved problems, and solutions which are easily searched by users. An idea of a use-case: lab A at some institution has a problem which impedes its progress and so it checks with idealab for a solution. Because the solution to this probelem hasn't been catalogued in an appropriately findable manner, lab A submits this problem to idealab.

Lab B has problem P. Lab B searches for a solution using standard methods. lab B finds lab A's record of problem P on ideaLab. Lab B registers another incidence of problem P.

IdeaLab has a system...

\section*{2015.08.27at22:53}
\subsection*{\#invention, \#softwareInScience, \#hackingForScience, \#ideaLab}
{\bf Continued from 2015.08.26at15:18}
IdeaLab has a system to rank ideas relative to one another. An idea's rank is a function that increases with the number of users who had this idea. Rank increases with the number of viewers it has.

Think of additional variables that rank varies with. The single most important task in creating ideaLab is to make idea browsing and discovery of ideas work well. IdeaLab is a great way to bring ideas of how to help the human endeavor to the layperson.

Suppose Bob the bricklayer has an idea which helps Portland Cement make a better cement. If only there was a framework for the sharing of ideas to be witnessed and paid for as a percentage of calculated increased profit. Players are motivated to pay handsomely for ideas to encourage new ideas. This an alternative to a conventional research and development department,

IdeaLab's job in this is:  provide the easy access to the hosted ideas, problems, and solutions.[I wanted to add more here about idealab and I wanna work on the idealab site as a web-app. Make this as a ruby, php, or python site.]

\section*{2015.08.31at14:35}
\subsection*{\#journal, \#DIY-ence, \#ideaLab}
People can run independent studies and experiments by throwing food to a group of ducks and recording the responses of the groups. When observing a small group, it seems to observe whether there is altruism displayed. For example, when a large clump of bread is thrown to the group, does the group let all members get some part of the clump or do the dominant ducks hoard the resource?

I want to read "open access" journals. This will keep me up on subject matters' current evolution.

\section*{2015.09.15at20:15}
\subsection*{\#journal}
I find myself wondering why are kthese prople the way they are. This is about the most ridiculous thing Ive seen. I have finally made myself come out and write and I actually am thinking "this is all I can come up with."

Language is a beautiful "arbitrary," as in "it was decided," and "mysterious," as in "what things can a study of this thing[, language,] tell us about nature. It beautiful that a finite set of symbols and rules can describe such a seemingly infinite set of things. This is a beauty worth exploring.

\subsection*{lyrics}
She's so sweet but I'm just me.
She never told me what to be.
If I am wrong then that's me and that's all that we'll be, but I love her.
Then I'm gone and that's all that I'll see of her.

What's there to love when she's gone?
I can't say what's not there.
I'm here.
I'll love me.
I love me.
I love to be
everything
never mean
know what I mean?

\section*{2015.10.09at10:15}
\subsection*{ \#schedule \#ideaConsulting }
I have decided to fulfill my dreams with the time scheduled for sleeping. This means I will read interesting and promising books with this time. I will write my thoughts and video record journals when necessary because typing can't keep up with the rate of my thoughts.

On that note, I should start playing with a typing tutor. I want to find a good one. Back to a thought from the previous paragraph, Google could make software for video journaling.

When thinking about the idea of an unaffiliated, freelancing individual having ideas that benefit a company, I think I should pioneer an industry I dub idea consulting.

\section*{ 2016.01.05at14:28 }
\subsection*{ \#inDefenseOfPeace }

I have not seen a disagreement that I thought was worth lives. Does fighting make a person tough? Forfeiting thinking for fighting appears to turn righteousness into wrongfulness. Can we think of a better way?

What if we use the money, energy, and passion currently dedicated to an ironically named Department of Defense to find appropriate resolutions to problems? What are you willing to spend to find resolutions which don't cost lives to raise hate?

Peaceful resolutions may be hard to find. People commend soldiers for participating in fights we all fund. What if people spent the same money and lives on clever diplomacy that avoids wars? What if we use love to spread love?

One idea is to reward diplomacy. What if there was a system to which any person can contribute diplomatic ideas and be rewarded for bringing a good idea? What if people could work together to find an elusive peaceful resolution?

Wikipedia's Software Industry article reads "according to industry analyst Gartner, the size of the worldwide software industry in 2013 was US\$407.3 billion." Even in this well funded industry, many of the most effective tools are created and actively maintained by thousands of volunteers. I am talking about open-sourced software.

The term opensource refers to software whose source-code is freely editable. This allows any person to contribute a resolution to any issue. In this way, the opensource community offers competitive solutions which are licensed to remain freely editable by posterity.

Can opensource show us how to collaborate on ideas for peaceful resolutions? Opensource provides solutions which competed with with the leading products of a US\$400 billion industry with notable successes. Opensource software dominates the market for web servers and secure cheap personal computing playgrounds.

\section*{ 2016.02.02at22:13 }
\subsection*{ \#mathematicalBreadcrumbs }
I hope that each who is in wonder at the majesties of either math or science one day realizes the beauty that complex exponentials store the arbitrary but beatifully succinct connection between algebra and rotations.

\section*{ 2016.02.06at18:34 }
\subsection*{ financialPlannery investmentStrategy }
Plan to save money at the rate $R$ dollars per time period(e.g. year, month, week, day,..). Put money away in savings in quantity $S$. Purchase in quantity $P$ for repairing the house with top quality fixes that increase equity or other acts which increase assets.

The killer idea here is to count asset-increasing-work as money saved in case saving everything over a reasonable spending limit doesn't work. This is, however, the preferred saving plan: 
\begin{itemize}
\item spend within spending limit $L$;
\item let $E$ be the amount earned during the period;
\item let $J$ be the current project, E.G. 
\item $E = $
\end{itemize}

With these definitions let $R=S+P$. 

\section*{ 2016.02.14at23:38 }
\subsection*{ addictedToLearning }
I want to be someone who is addicted to learning. I want to meet someone who is addicted to learning. I specifically want to meet someone who is on trajectory to know physics and computer science.

\section*{  2016.02.27at16:47 }
\subsection*{  \#stolenJournalingTime }
It strikes me while trying to journal right now that I need to script the journaling process better. I need to make kydate return a string with no newlines in it. I think echo wraps a cstring in newlines.


\section*{ 2016.03.19at03:14 }
\subsection*{ testJournal.txt fascinatingTopic alsoFascinating }
This is truly some kind of shit. I cant type on my chromebook becasue there was this really cool idea somebody had to be clever by making the trackpad's scrollwheel (the... I had just forgotten to run my trackpad configuration tool after restarting the chromebook.

\#iRemember

\section*{ 2016.03.19at12:19 }
\subsection*{ \#hashtags \#reviewingLastEntry }
hahah, looking at the above, I forgot what I had remembered momentarily. At least I see it as funny, bordering on hilarious. I think this is a great way to handle something that might be heartbreaking if handled differently. This is a good example of how to emotionally handle lost wonderful ideas. 

\section*{ 2016.03.27at22:36 }
\subsection*{ millionDollarIdea }
I want to start a music hosting company with a stated mission to "make money in a way that doesn't hurd the freedom of the music."

Business plan:
Host a library of music, books, guides, tutorials, and other works that people want to self-publish. Hosted parties may choose a license for their works. Perhaps model monetization openly after github: users must pay for storage of their work to license the work as closed-to-derivative-works. 

\section*{ 2016.03.27at23:21 }
\subsection*{ whatDidILearnFromPhysics aLetterToThoseComingAfter }
I learned that approximations which describe a phenomenon to first order are a fine starting place to build an understaning of the phenomenon. I found from this that observations of first oder behavior are a fine place to begin observation. The lesson of the power of the first orders I've learned through many related lessons.

I learned that "limiting cases are a physicist's bread and butter," said some physics professor at OSU. I have found several situations where this statement resounds in my memory. One application I have developed great affection for is in deciding how to make a first guess at a model for observed phenomena.

Using these two teachings together, one can make useful guesses at the behavior of phenomena. From a first order observation, one knows the limiting behavior of $f(x)$ for both decreasing and increasing $x$. This is considered an observation of the behavior of $f(x)$ to first order in $x$.

Notice I said a lot of words to describe what has been observed. All these words have meaning. It is enlightening to know the significance of every single word when a person speaks purposefully.

There are a calculated number of degrees of freedom removed from a general function one may observe to first order in one of its independent variables. Some examples of the behavior of $f(x)$ to first order in $x$ follow.

\begin{itemize}
\item The value of $f$ may increase as the variable $x$ increases:$$f(x_2) > f(x_1) \Leftrightarrow x_2 > x_1.$$ This is said "$f$ varies directly with $x$."
\item The value of $f$ may decrease as the variable $x$ decreases:$$f(x_2) < f(x_1) \Leftrightarrow x_2 < x_1.$$ This is said "f varies inversely with $x$."
\item The value may behave stochastically, chaotically, but always functional. Functional because $f(x)$ always has the same value. Every day of the week.

One way to say this logically and in a way which requires and might teach some, mathematical vocabulary:
"The function $f()$ is a bijective function. $f(x_0)$ is a single value for a single value of $x_0$."(I didn't remember what I started writing...I just came back and it's now time for another quick \\openinglst\$journal\\closedalsting-ing.
\end{itemize}

\section*{ 2016.03.31at00:21 }
\subsection*{\#{million dollar idea}}
Talk to Ryan [Parmeter, 2016.03.31at06:14]about google getting behind the command line for usability. People who use the command line fall in love with computers. A couple different ways of going about this are to deliver hardware along with drivers.

This could be called a platform(here's a golden idea for a product name: a springboard).

\section*{ 2016.03.31at23:00 }
\subsection*{\#{stream of consciousness} \#anneFilson } 
Write to Anne Filson to schedule some coffee at lunch-time. I want to write a note asking for a coffee and a conversation. Some things I want to discuss are listed here.

I want to mention gut flora possibly needing the bacterial byproducts of fermentation. I want Anne's input about the data pertaining to the hominin diet. What do we know about homini and human diet throughout the evolution of our species.

This could be a beneficial topic for someone trying to get his message of veganism out to a large audience. This could also be a publishable work: people possibly used to fill their bellies beyond full to both get bacterial growth in the intestines. This thought occurred to me when I noticed after a day of overeating that my belch tasted like alcohol(This was within a month of falling off my bicycle and breaking out my front teeth following a night of drinking and swearing off drinking (except for Rich Tehan's birthday in 2016)).

\section*{ 2016.04.01at11:58 }
\subsection*{\#\{million dollar idea\}}
Discuss the following ideas I've had with Ryan Parmeter, who is a friend from the hesper.net building that Ryan Schrink ran and drained the rainwater out of the ceiling's insulation by pushing a broom up into the plastic layer covering the bottom of the insulation-packs(it was packs similar to, if not exactly, interstud insulation from exterior wall insulation).

Collaborate with Google in a way that gives me more power as an "inventor" to get my ideas out so the ideas may better benefit humanity. I have four ideas now that I want to write here before I forget them for Google, including the corporation-and-inventor-collaboration mentioned above.  

\section*{ 2016.04.02at15:14 }
\subsection*{\#millionDollarIdea}
The items I wanted to address with Ryan Parmeter to present to Google with a witness are those following. I want to have a witnessed meeting with executives from Google to present ideas for Google to use, cutting me some originator's percentage of the profits for the first few years. Ryan could serve as my witness.

Google could make a device and the software to provide functioality like the nest home-control device. This device would serve as an upgrade to older vehicles to provide the features of newer cars as well as any feature that people dream up. Example features are navigation systems, phone interface systems, backup cameras, and antitheft features. This serves as something the EPA might be interested in endorsing because upgrading cars' features would presumably reduce waste by slowing the rate at which cars are disposed of.

Google could offer a feature I call a "sensible text-size" in chrome to detect a main text-display-elements of webpages. Sensible text-size uses the number of characters per line known to be optimal from typesetting. I think this value is 55 to 65 characters per line.

Once this happens, I guess that others would make their implementation of this feature. I imagine that consequent to browsers and power users implementing this feature web design best practices would evolve to evolve to cater to this feature. This feature would allow for command-line-browsing, which propogates the UNIX paradigm of embracing a text-stream as the perfect data-structure.

The efficiency gains in computer-use from implementing the UNIX philosophy segues nicely into the final idea which I am most excited about: Google supplies a chromebook that is made to host crouton. Preferably, Google supplies the drivers for use in a standalone linux installation, so that users don't have to use up resources keeping ChromeOS loaded. Google could mandate that hardware manufacturers supply drivers for their devices to the open-source-community. 

\section*{ 2016.04.11at07:45 }
\subsection*{millionDollarIdea}
Google could lead the way in voice recognition technologies by using veiwers' corrections to Youtube's auto-generated captions. This would also be a wonderful benefit to those suffering from a disability which puts them in need of the captions or even transcripts from a video. It is not terrible wild to imagine a system which could give captions in realtime.

\section*{ 2016.04.19at23:45 }
\subsection*{streamOfConsciousness}
I finally made myselfsit down and write. thoughts are thinking much faster than the words get set in type. I need to do typing eercises to rectify this. Now I've had the idea that I've had before: I can journal when I have a decent thought during the time I am 

\section*{ 2016.04.22at13:26 }
\subsection*{restaurantBestPractices}
Restaurants that offer outdoor seating that might get rained on could let customers dry off their chosen seats with a pile of laundry set out at a self-service station which also offers flavoring mixes, self-busing bins, utensils, and garbage.

\section*{ 2016.04.22at13:32 }
\subsection*{amusumentParkBusinessPlan}
I want to create a restaurant attached dog park attached to human-powered amusement rides attached to a large host of other attractions. The place should have a part of the restaurant where the people can watch their children and dogs playing in their respective playgrounds and in each others' playground. 

\section*{ 2016.04.20at22:16 }
\subsection*{journalingWhileHigh}
I propose the following activity to myself. My proposal will will start to sound something like the beginning of a book. That last sentence just seems like a hint and so I wonder if I'll wonder at some point in my life when I first realized writing will be the point of my life's work.

I shall start walking every day; getting envigorating exercise multiple times a week. I don't need hobbies to fill up my time because I have the obligation to myself of having thought of living an experiment in health as an example to my fellow people.

People are not humans. I mean this as people are not the blood, flesh, and bone that comprises the physical body of a person. Here I think person as the singular form of people. People are the tiny, apparently singular spark in the brain that is, nevertheless the product of a concerted effort on a statistical number of cells. I make the following broadly generalized statement as a guess at something possibly worth investigating at a later time: the mystery of life can, following from the last sentence be said to be shadowed from our understanding by statistics.

I am in the middle of writing a letter, of course on Facebook, to the physics girl proposing we make educational videos on a host of topics for free to help all people's existence. I am arguing that this will benefit in...I want to revisit several issues in this entry. 

\section*{ 2016.05.02at00:47 }
\subsection*{ lifePlan research academicPlan }
I want to do a project which revolutionizes education in physics and other subjects. It is a jupyter notebook with embedded video built right in. Mining the data from this to further physics education research is a promising motivation of this research. This environment can clearly be used for any other subject.

The multimedia learning resource can be any medium, whether slides with audio, only audio, or video. I can take this idea to professors Roundy, Manogue, Qiu, 

\section*{2016.05.03at21:13}
\subsection*{iWantToWritebABook}

\section*{ 2016.05.13at23:05 }
\subsection*{ gottaTakeThisStreamDown }
[2016.01.15@21:39 oops, I guess I missed that one...bummer, dude:-[]


\section*{ 2016.06.15at21:41 }
\subsection*{ artificialScience thatLastOneIsFunnyBut getKyleInScience then getKyleNScience }
I have an idea I want to pursue.
I suspect Roundy could be helpful in this.
I want to develop a way for humans to get more intimate with the beauty that computers can provide through:
\begin{enumerate}
\item interacting with a beautiful operation system;
\item creating a logical entity as a pet with that operating system that is everywhere and can do anything between, alongside and through its host(computer)'s inputs and outputs;

\section*{ 2016.06.15at21:54 }
\subsection*{ ideasForScientificInvestigation }
I want to investigate something I will describe starting with the particular example whose consideration led me to think of the object of the investigation I hope to pursue.
Let a hollow glass cylinder be filled with a given liquid.
Orient the cylinder so its axis is vertical and close the top of its hole.

This liquid has an interesting property that appears on consideration to be characteristic of the liquid's mass density per unit of force of interparticle attraction.

That property is the distance which a unit mass will fall under its own weight.

\section*{ 2016.05.18at23:30 }
\subsection*{iNight(lyRoutine)}

\section*{ 2016.05.19at08:55 }
\subsection*{coffeeCommuteAllenBrosOnMonroe}
I would like to take the idea of putting linux on a chromebook to Google. The hardware supports me sitting in the sun and type 16 hours on a single charge. With a linux backend, this blows away anything currently sold as a laptop.

\section*{ 2016.05.19at18:41 }
\subsection*{streetWisdom}
A friendly street-musician named Daniel Faulk remarked a very true statement: "[it is easier to be creative when in that place where trying to simultaneously perform two tasks which are each difficult on its own.]" What he said was actually literally closer to "[Soloing is hard. Playing rhythm is hard. When you're in the place between the two you're more creative.]" He said this while telling me he can't do both at the same at the same time.

This leads me to the thought: what if people's burst of creativity when first learning to sing and play or do any two mentally engaging activitiies at the same time.

\section*{ 2016.05.20at11:58 }
\subsection*{physicsIsFun mathematicalFoundationOfExponentialsInStatMech}
Professor Matt Graham at OSU told me something that I want to take down. He said this while teaching ph641:statistical thermophysics that I attended the year after graduating OSU while I was working a coding contract for the EPA. He said that the ubiquitous exponential in statistical thermophyics arises from an ansatz: Stirling's Approximation.

\section*{ 2016.05.20at12:05 }
\subsection*{statMech}
It seems to me at first consideration that it may be the uncoupling of dependence of the quantum states of neighboring water molecules which produces the greater opacity of water in the state of steam than in the state of liquid.

\section*{ 2016.05.25at11:26 }
\subsection*{lateWorkdayMorningRoutine}
I could do wonderful things with a position at Google. I have this idea that they could partner with the opensourced community. As developers in this community form startup interests, Google could selectively partner with them.

This model doesn't have to dominate Google's monetization efforts. This Opensourced-Partnership-Program(OPP?) could nicely complement Google's current marketing technique. Further, OPP provides dan example of Google's old mantra: "don't be evil."

\section*{ 2016.05.26at07:34 }
\subsection*{thursdayMorningCoffee}


\section*{ 2016.05.26at07:51 }
\subsection*{currentLifePlan lifeOneDayPerTime oneDay@aTime oneDayAtATime}
I made a schedule for my work weeks: wake at 5am to beat the sun to hitting the ground; get to work by 6am Monday through Thursday and every other Friday; work until 10:30am Monday through Thursday and every other Friday; take a lunch break, run errands, and journal with remaining time from 10:30am to 1pm; work from 1pm to 5:30pm Monday through Thursday; work from 1pm to 4:30pm on every other Friday.

This schedule works me 40hours every week when weeks are split at noon on Friday. I work 9 hours Monday through Thursday. I work eight hours one Friday. I don't work the next friday.

The plan is to do my nightly routine by 9pm. Nightly routine is meant to include a final smoke-session. I plan to have another smoke-session if aid is needed for sleep. Either way, the plan holds to be in bed with lights out and teeth brushed since my last smoke by 11pm.

\section*{ 2016.05.29at19:32 }
\subsection*{dinnerAtSterlings}
I was remembering that Portugal has "decriminalized" all drugs [Todd asked me to put electronics away. 2016.05.31@11:00:00]

\section*{ 2016.05.31at11:07 }
\subsection*{personalJournal}
I have stopped into Allen Bros on 26th and Arnold before going to work. I just got done with a partial root canal with the dentist at the Corvalis Gentle Dental. He stopped part of the way through the root canal because he wasn't confident proceeding. His office referred me back to the Gentle Dental in Salem called Gentle Dental at Idlewood.

Idlewood is the office of Wayne Van de Graaf, the oral surgeon who reimplanted my front teeth. This was the morning that I had crashed at around 1am. Dr. Treloar at Corvallis's Good Samaritan ER had originally placed my teeth back in their sockets quickly in hopes of the teeth surviving.

It turned out that Dr. Treloar had replaced the teeth in the incorrect positions and directions. I feel grateful that Treloar had done such a caring and careful job as he did in helping me deal with the consequences of my morning's drunken bike ride from the bar. In order to do my part, I resolved to stop drinking and have been successful other than a shot at Richard Tehan's birthday and a sip of beer at Amanda Tyree's bidding while sitting on Bastendorf Beach watching unrideably small surf with her and Sterling.

I believe I have journaled this before, but I want to say this again: I don't want to drink. Abstinence from alcohol is motivated by a desire to maintain and imrove my health, a desire to be more productive than I am when I drink, and for safety. Falling off my bike and knocking my teeth out was not my first lesson about safety and drinking. I plan to make this incident my last.

\section*{ 2016.06.01at08:11 }
\subsection*{portlandForTheEndodontist}

Below is a table that revisits the weekly plan that was made in a previous post.

\begin{table}
\caption{\label{tab:2016.06.01WeeklyAndDailyTimeTable}Weekly and Daily Time-Table}
\begin{tabular}{|r|r|l|}
\hline
Hours/Week&Hours/Day&Activity\\
\hline
7&1&hygiene\\
7&1&workout\\
49&12$\times$4+1&work\\
14&2&social and play time\\
42&6&sleep\\
21&3&study\\
28&4&journal and invent\\
\hline
\end{tabular}\end{table}

\begin{table}
\caption{\label{tab:2016.06.01DailySchedule}Daily Schedule}
\begin{tabular}{|r|r|r|r|r|r|r|r|r|r|r|r|r|r|r|r|r|r|r|r|r|r|r|r|}
\hline
1&2&3&4&5&6&7&8&9&10&11&12&13&14&15&16&17&18&19&20&21&22&23&24\\
\hline
&&&4:30 wake&5:00 commute\\5:20 workout\\5:40 shower&6 work&&&&10:30 lunch&&&13 work&&&&5:30 leave work&&&&&&&\\
\hline
\end{tabular}\end{table}

\begin{enumerate}
\item 6:00 - wake
\item 6:20 - commute
\item 6:40 - workout
\item 7:20 - shave and shower
\item 7:45 - eat
\item 8:00 - work
\item 11:00 - lunch
\item 11:30 - work
\item 2:30 - walk
\item 3:00 - work
\item 6:00 - get off
\end{enumerate}

\section*{ 2016.06.11at14:11 }
\subsection*{responsiblePackaging suggestionsToIndustry pleaseProduceReusableContainers youAlreadyPackageThings}
What would happen if humans began to create things in a nondisposable fashion? What if we began creating packaging for groceries that could be washed and reused as household Ziploc-bags or for fruit and goods at another market. Potatos could be packaged in bags that could work as sacks to chuck random camping or gardening items into.

If these things are reusable and they are made of high enough quality that people wanted to keep them around and reuse them for stuff then it would be good PR for the company.

\section*{ 2016.06.12at14:55 }
\subsection*{softwareIdea InterfaceIdeaReally}
I want to make a multimedia editing visual interface that displays a waveform for the audiotracks zoomed to any region of interest. Users could pinch to zoom the region of interest, lay layers of effects which are visible in a tagged region specified by a brace. The sound filters should be able to be added, ordered and removed atomically.

Users can play the visible region or a tagged subregion. Tagging of subregions can be similar to tagging regions where effects are applied. Effects that start and end at the same time should be tagged together.

\section*{ 2016.06.17at19:52 }
\subsection*{ideaToEndWoefulStateOfPublicHealthPoliticalCorrectness}
I want to speak with Allan Thornhill, the newly appointed chief of the Western Ecology Division(WED). I could speak with with my immediate supervisor, Allen Brookes. Tertiarily, I could speak with the contractor in charge of the project with the fegs-dashboard, Kirsten Winters. Quaternarily, I could speak with one of my cubicle-mates, either Colleen Barr, Patrick Winger, or Vladimir Pedovic.

Really, I should speak to everyone I can. Being strategic about whom I tell my good ideas is more selfish than I like to perceive myself. It is a dance, as, at this moment, it feels like everything is, a dance. This dance is balancing economy of effort and sharing my ideas as much as possible.

I want to propose the idea to make it politically incorrect to feed people animal-products in the same way it is politically incorrect to exclude people with disabilities. This idea comes to me from being outraged that general employee-funds are being spent to provide animal products for food at an EPA function.

It is my opinion that the WED needs to make a bold move to act outside the connotations of its scope. By this I mean that WED needs to concern itself with human endeavor. People are part of the ecology of their region. People are animals. It is pompous, arrogant, conceited, flawed, and dangerous to not model human animals as part of ecosystems.

We need to address the conflict of interest the USDA has with disseminating health information. 

The study results are in. The evidence is published. Why does public-policy allow people's education in nutrition to be led by agricultural interests.

\section*{ 2016.06.17at20:18 }
\subsection*{emotionalOscillationIsAThingThatDescribesCausalChemicalPathways}

\section*{ 2016.06.17at20:21 }
\subsection*{calculateTheLengthOfAPlaneThatAppearsAsLongAsAQuarterInchArm'sLengthFromMyEye}

\section*{ 2016.06.26at15:22 }
\subsection*{somethingToRemember conversationWithMom}
Oops,i forgot.

\section*{ 2016.07.03at12:08 }
\subsection*{myLifesWork}
Being Nice is my life's work.

\section*{ 2016.07.03at12:12 }
\subsection*{theOnlyCorrectPronunciation}
The only correct pronunciation is the one the speaker judges to be the best.


\section*{ 2016.07.17at22:17 }
\subsection*{ #writtenWithBlahScript }
iHaveSpentMyLifeOnThis inventSomeClassNamedAbstractics abstractricsBecomesAbstractrix

\section*{ 2016.07.17at22:19 }
\subsection*{LaTeX}
Sometimes my journal is written in the mode of me telling myself to do things. The following will be a brain dump of thoughts I've recently entertained. Note to self: get your parser ready.

I want to play with evolution by throwing out every kind of flower- and delicious edible fruit seeds at parks. We put money into homeless shelters when we could just grow edible plants in public spaces. Caring for society's people, stated people in general because rich and poor people alike might forage the local spaces, could become an endeavor that private groups and individuals undertake by leaving community-produce-trees by the side of the road.

\section*{ 2016.07.31at20:03 }
\subsection*{ #writtenWithBlahScript }
deleteme testingBlahScriptAfterLongTimeAwaySoNoKowIfWorks iAm

\section*{ 2016.07.31at20:19 }
\subsection*{ #writtenWithBlahScript }
imReallyGonnaUseMyLaptopToRecordMeInMyLivingRoom
||||||| merged common ancestors
=======
section*{Journal Stream Started 2015.08.21}

\subsection*{\#loveLife}
This is some stuff. The stuff is words, really. The amazing thing is I get to string them together however I please.

I am so in love because of who Amanda Chadderdon is and how lovingly Amanda has behaved toward me. I am scared to tell her this because I think it may scare her off. I don't know what to do.

I say "I'm not sure what to do" as though, and I want to joke that "I \textbf{AM} sure I'll be happy however my life pans out." but I say "I'm not sure" as though...I LOST THE THOUGHT OVER THE COURSE OF HOURS...

\subsection*{\#lifePlan}
I wanna journal about what I'll do with my life. I have come to the conclusion that...CONTINUED ON NEXT

\section*{2015.08.21at06:57}
\subsection*{\#lifePlan} CONTINUED FROM YESTERDAY'S THOUGHT:
"I have come to the conclusion that..." I should create a vegan cheese while I'm looking for work and working off student loans as a cab-driver. In order to plan and verify feasability, I calculate some basic time requirements imposed by all the things I want to do(after this table, a current plan is to be kept in tabulated form in weeklyTimeTable.tex):

\begin{table}
\caption{\label{tab:originalWeeklyAndDailyTimeTable}Weekly and Daily Time-Table}
\begin{tabular}{|r|r|l|}
\hline
Hours/Week&Hours/Day&Activity\\
\hline
7&1&hygiene\\
7&1&workout\\
49&12$\times$4+1&drive a cab\\
14&2&social and play time\\
42&6&sleep\\
21&3&study graduate physics courses\\
28&4&journal and invent\\
\hline
\end{tabular}\end{table}

This gives $7 \text{days} \times 24 \text{hours} = 168$ hours per week, accounting for the all hours in a week. I intend to investigate using food, sleep, and drugs to augment my productivity in learning and inventing. Here, inventing can be broadly interpretted.

\subsection*{\#invention}
Invention could mean creating a vegan cheese by chemical synthesizing a substrate with the desired percentages and types of fat, protein, and carbohydrate. Invention could mean designing a belt for runners to store an empty or full water bladder in, as well as shopping bags, a phone, pens, keys, clean and dry clothes, and other items. Invention could mean designing a pair of gloves with all the things listed for the belt attached as well as carrying pouches to provide runners with an ergonomic upper-body workout while running.

\section*{2015.08.26at15:18}
\subsection*{\#invention, \#softwareInScience, \#hackingForScience}
Make idealab: a database of possibly networked ideas, unsolved problems, and solutions which are easily searched by users. An idea of a use-case: lab A at some institution has a problem which impedes its progress and so it checks with idealab for a solution. Because the solution to this probelem hasn't been catalogued in an appropriately findable manner, lab A submits this problem to idealab.

Lab B has problem P. Lab B searches for a solution using standard methods. lab B finds lab A's record of problem P on ideaLab. Lab B registers another incidence of problem P.

IdeaLab has a system...

\section*{2015.08.27at22:53}
\subsection*{\#invention, \#softwareInScience, \#hackingForScience, \#ideaLab}
{/bf Continued from 2015.08.26at15:18}
IdeaLab has a system to rank ideas relative to one another. An idea's rank is a function that increases with the number of users who had this idea. Rank increases with the number of viewers it has.

Think of additional variables that rank varies with. The single most important task in creating ideaLab is to make idea browsing and discovery of ideas work well. IdeaLab is a great way to bring ideas of how to help the human endeavor to the layperson.

Suppose Bob the bricklayer has an idea which helps Portland Cement make a better cement. If only there was a framework for the sharing of ideas to be witnessed and paid for as a percentage of calculated increased profit. Players are motivated to pay handsomely for ideas to encourage new ideas. This an alternative to a conventional research and development department,

IdeaLab's job in this is:  provide the easy access to the hosted ideas, problems, and solutions.[I wanted to add more here about idealab and I wanna work on the idealab site as a web-app. Make this as a ruby, php, or python site.]

\section*{2015.08.31at14:35}
\subsection*{\#journal, \#DIY-ence, \#ideaLab}
People can run independent studies and experiments by throwing food to a group of ducks and recording the responses of the groups. When observing a small group, it seems to observe whether there is altruism displayed. For example, when a large clump of bread is thrown to the group, does the group let all members get some part of the clump or do the dominant ducks hoard the resource?

I want to read "open access" journals. This will keep me up on subject matters' current evolution.

\section*{2015.09.15at20:15}
\subsection*{\#journal}
I find myself wondering why are kthese prople the way they are. This is about the most ridiculous thing Ive seen. I have finally made myself come out and write and I actually am thinking "this is all I can come up with."

Language is a beautiful "arbitrary," as in "it was decided," and "mysterious," as in "what things can a study of this thing[, language,] tell us about nature. It beautiful that a finite set of symbols and rules can describe such a seemingly infinite set of things. This is a beauty worth exploring.

\subsection*{lyrics}
She's so sweet but I'm just me.
She never told me what to be.
If I am wrong then that's me and that's all that we'll be, but I love her.
Then I'm gone and that's all that I'll see of her.

What's there to love when she's gone?
I can't say what's not there.
I'm here.
I'll love me.
I love me.
I love to be
everything
never mean
know what I mean?

\section*{2015.10.09at10:15}
\subsection*{ \#schedule \#ideaConsulting }
I have decided to fulfill my dreams with the time scheduled for sleeping. This means I will read interesting and promising books with this time. I will write my thoughts and video record journals when necessary because typing can't keep up with the rate of my thoughts.

On that note, I should start playing with a typing tutor. I want to find a good one. Back to a thought from the previous paragraph, Google could make software for video journaling.

When thinking about the idea of an unaffiliated, freelancing individual having ideas that benefit a company, I think I should pioneer an industry I dub idea consulting.

\section*{ 2016.01.05at14:28 }
\subsection*{ \#inDefenseOfPeace }

I have not seen a disagreement that I thought was worth lives. Does fighting make a person tough? Forfeiting thinking for fighting appears to turn righteousness into wrongfulness. Can we think of a better way?

What if we use the money, energy, and passion currently dedicated to an ironically named Department of Defense to find appropriate resolutions to problems? What are you willing to spend to find resolutions which don't cost lives to raise hate?

Peaceful resolutions may be hard to find. People commend soldiers for participating in fights we all fund. What if people spent the same money and lives on clever diplomacy that avoids wars? What if we use love to spread love?

One idea is to reward diplomacy. What if there was a system to which any person can contribute diplomatic ideas and be rewarded for bringing a good idea? What if people could work together to find an elusive peaceful resolution?

Wikipedia's Software Industry article reads "according to industry analyst Gartner, the size of the worldwide software industry in 2013 was US\$407.3 billion." Even in this well funded industry, many of the most effective tools are created and actively maintained by thousands of volunteers. I am talking about open-sourced software.

The term opensource refers to software whose source-code is freely editable. This allows any person to contribute a resolution to any issue. In this way, the opensource community offers competitive solutions which are licensed to remain freely editable by posterity.

Can opensource show us how to collaborate on ideas for peaceful resolutions? Opensource provides solutions which competed with with the leading products of a US\$400 billion industry with notable successes. Opensource software dominates the market for web servers and secure cheap personal computing playgrounds.

\section*{ 2016.02.02at22:13 }
\subsection*{ \#mathematicalBreadcrumbs }
I hope that each who is in wonder at the majesties of either math or science one day realizes the beauty that complex exponentials store the arbitrary but beatifully succinct connection between algebra and rotations.

\section*{ 2016.02.06at18:34 }
\subsection*{ financialPlannery investmentStrategy }
Plan to save money at the rate $R$ dollars per time period(e.g. year, month, week, day,..). Put money away in savings in quantity $S$. Purchase in quantity $P$ for repairing the house with top quality fixes that increase equity or other acts which increase assets.

The killer idea here is to count asset-increasing-work as money saved in case saving everything over a reasonable spending limit doesn't work. This is, however, the preferred saving plan: 
\begin{itemize}
\item spend within spending limit $L$;
\item let $E$ be the amount earned during the period;
\item let $J$ be the current project, E.G. 
\item $E = $
\end{itemize}

With these definitions let $R=S+P$. 

\section*{ 2016.02.14at23:38 }
\subsection*{ addictedToLearning }
I want to be someone who is addicted to learning. I want to meet someone who is addicted to learning. I specifically want to meet someone who is on trajectory to know physics and computer science.

\section*{  2016.02.27at16:47 }
\subsection*{  \#stolenJournalingTime }
It strikes me while trying to journal right now that I need to script the journaling process better. I need to make kydate return a string with no newlines in it. I think echo wraps a cstring in newlines.
 

\section*{ 2016.03.19at03:14 }
\subsection*{ testJournal.txt fascinatingTopic alsoFascinating }
This is truly some kind of shit. I cant type on my chromebook becasue there was this really cool idea somebody had to be clever by making the trackpad's scrollwheel (the... I had just forgotten to run my trackpad configuration tool after restarting the chromebook.

\#iRemember

\section*{ 2016.03.19at12:19 }
\subsection*{ \#hashtags \#reviewingLastEntry }
hahah, looking at the above, I forgot what I had remembered momentarily. At least I see it as funny, bordering on hilarious. I think this is a great way to handle something that might be heartbreaking if handled differently. This is a good example of how to emotionally handle lost wonderful ideas. 

\section*{ 2016.03.27at22:36 }
\subsection*{ millionDollarIdea }
I want to start a music hosting company with a stated mission to "make money in a way that doesn't hurd the freedom of the music."

Business plan:
Host a library of music, books, guides, tutorials, and other works that people want to self-publish. Hosted parties may choose a license for their works. Perhaps model monetization openly after github: users must pay for storage of their work to license the work as closed-to-derivative-works. 

\section*{ 2016.03.27at23:21 }
\subsection*{ whatDidILearnFromPhysics aLetterToThoseComingAfter }
I learned that approximations which describe a phenomenon to first order are a fine starting place to build an understaning of the phenomenon. I found from this that observations of first oder behavior are a fine place to begin observation. The lesson of the power of the first orders I've learned through many related lessons.

I learned that "limiting cases are a physicist's bread and butter," said some physics professor at OSU. I have found several situations where this statement resounds in my memory. One application I have developed great affection for is in deciding how to make a first guess at a model for observed phenomena.

Using these two teachings together, one can make useful guesses at the behavior of phenomena. From a first order observation, one knows the limiting behavior of $f(x)$ for both decreasing and increasing $x$. This is considered an observation of the behavior of $f(x)$ to first order in $x$.

Notice I said a lot of words to describe what has been observed. All these words have meaning. It is enlightening to know the significance of every single word when a person speaks purposefully.

There are a calculated number of degrees of freedom removed from a general function one may observe to first order in one of its independent variables. Some examples of the behavior of $f(x)$ to first order in $x$ follow.

\begin{itemize}
  \item The value of $f$ may increase as the variable $x$ increases:$$f(x_2) > f(x_1) \Leftrightarrow x_2 > x_1.$$ This is said "$f$ varies directly with $x$."
   \item The value of $f$ may decrease as the variable $x$ decreases:$$f(x_2) < f(x_1) \Leftrightarrow x_2 < x_1.$$ This is said "f varies inversely with $x$."
   \item The value may behave stochastically, chaotically, but always functional. Functional because $f(x)$ always has the same value. Every day of the week.

One way to say this logically and in a way which requires and might teach some, mathematical vocabulary:
"The function $f()$ is a bijective function. $f(x_0)$ is a single value for a single value of $x_0$."(I didn't remember what I started writing...I just came back and it's now time for another quick \\openinglst\$journal\\closedalsting-ing.
\end{itemize}

\section*{ 2016.03.31at00:21 }
\subsection*{\#{million dollar idea}}
Talk to Ryan [Parmeter, 2016.03.31at06:14]about google getting behind the command line for usability. People who use the command line fall in love with computers. A couple different ways of going about this are to deliver hardware along with drivers.

This could be called a platform(here's a golden idea for a product name: a springboard).

\section*{ 2016.03.31at23:00 }
\subsection*{\#{stream of consciousness} \#anneFilson } 
Write to Anne Filson to schedule some coffee at lunch-time. I want to write a note asking for a coffee and a conversation. Some things I want to discuss are listed here.

I want to mention gut flora possibly needing the bacterial byproducts of fermentation. I want Anne's input about the data pertaining to the hominin diet. What do we know about homini and human diet throughout the evolution of our species.

This could be a beneficial topic for someone trying to get his message of veganism out to a large audience. This could also be a publishable work: people possibly used to fill their bellies beyond full to both get bacterial growth in the intestines. This thought occurred to me when I noticed after a day of overeating that my belch tasted like alcohol(This was within a month of falling off my bicycle and breaking out my front teeth following a night of drinking and swearing off drinking (except for Rich Tehan's birthday in 2016)).

\section*{ 2016.04.01at11:58 }
\subsection*{\#\{million dollar idea\}}
Discuss the following ideas I've had with Ryan Parmeter, who is a friend from the hesper.net building that Ryan Schrink ran and drained the rainwater out of the ceiling's insulation by pushing a broom up into the plastic layer covering the bottom of the insulation-packs(it was packs similar to, if not exactly, interstud insulation from exterior wall insulation).

Collaborate with Google in a way that gives me more power as an "inventor" to get my ideas out so the ideas may better benefit humanity. I have four ideas now that I want to write here before I forget them for Google, including the corporation-and-inventor-collaboration mentioned above.  

\section*{ 2016.04.02at15:14 }
\subsection*{\#millionDollarIdea}
The items I wanted to address with Ryan Parmeter to present to Google with a witness are those following. I want to have a witnessed meeting with executives from Google to present ideas for Google to use, cutting me some originator's percentage of the profits for the first few years. Ryan could serve as my witness.

Google could make a device and the software to provide functioality like the nest home-control device. This device would serve as an upgrade to older vehicles to provide the features of newer cars as well as any feature that people dream up. Example features are navigation systems, phone interface systems, backup cameras, and antitheft features. This serves as something the EPA might be interested in endorsing because upgrading cars' features would presumably reduce waste by slowing the rate at which cars are disposed of.

Google could offer a feature I call a "sensible text-size" in chrome to detect a main text-display-elements of webpages. Sensible text-size uses the number of characters per line known to be optimal from typesetting. I think this value is 55 to 65 characters per line.

Once this happens, I guess that others would make their implementation of this feature. I imagine that consequent to browsers and power users implementing this feature web design best practices would evolve to evolve to cater to this feature. This feature would allow for command-line-browsing, which propogates the UNIX paradigm of embracing a text-stream as the perfect data-structure.

The efficiency gains in computer-use from implementing the UNIX philosophy segues nicely into the final idea which I am most excited about: Google supplies a chromebook that is made to host crouton. Preferably, Google supplies the drivers for use in a standalone linux installation, so that users don't have to use up resources keeping ChromeOS loaded. Google could mandate that hardware manufacturers supply drivers for their devices to the open-source-community. 

\section*{ 2016.04.11at07:45 }
\subsection*{millionDollarIdea}
Google could lead the way in voice recognition technologies by using veiwers' corrections to Youtube's auto-generated captions. This would also be a wonderful benefit to those suffering from a disability which puts them in need of the captions or even transcripts from a video. It is not terrible wild to imagine a system which could give captions in realtime.

\section*{ 2016.04.19at23:45 }
\subsection*{streamOfConsciousness}
I finally made myselfsit down and write. thoughts are thinking much faster than the words get set in type. I need to do typing eercises to rectify this. Now I've had the idea that I've had before: I can journal when I have a decent thought during the time I am 

\section*{ 2016.04.22at13:26 }
\subsection*{restaurantBestPractices}
Restaurants that offer outdoor seating that might get rained on could let customers dry off their chosen seats with a pile of laundry set out at a self-service station which also offers flavoring mixes, self-busing bins, utensils, and garbage.

\section*{ 2016.04.22at13:32 }
\subsection*{amusumentParkBusinessPlan}
I want to create a restaurant attached dog park attached to human-powered amusement rides attached to a large host of other attractions. The place should have a part of the restaurant where the people can watch their children and dogs playing in their respective playgrounds and in each others' playground. 

\section*{ 2016.04.20at22:16 }
\subsection*{journalingWhileHigh}
I propose the following activity to myself. My proposal will will start to sound something like the beginning of a book. That last sentence just seems like a hint and so I wonder if I'll wonder at some point in my life when I first realized writing will be the point of my life's work.

I shall start walking every day; getting envigorating exercise multiple times a week. I don't need hobbies to fill up my time because I have the obligation to myself of having thought of living an experiment in health as an example to my fellow people.

People are not humans. I mean this as people are not the blood, flesh, and bone that comprises the physical body of a person. Here I think person as the singular form of people. People are the tiny, apparently singular spark in the brain that is, nevertheless the product of a concerted effort on a statistical number of cells. I make the following broadly generalized statement as a guess at something possibly worth investigating at a later time: the mystery of life can, following from the last sentence be said to be shadowed from our understanding by statistics.

I am in the middle of writing a letter, of course on Facebook, to the physics girl proposing we make educational videos on a host of topics for free to help all people's existence. I am arguing that this will benefit in...I want to revisit several issues in this entry. 

\section*{ 2016.05.02at00:47 }
The multimedia learning resource can be any medium, whether slides with audio, only audio, or video. I can take this idea to professors Roundy, Manogue, Qiu, 
<<<<<<< HEAD

\section*{ 2016.05.13at23:05 }
\subsection*{ gottaTakeThisStreamDown }
[2016.01.15@21:39 oops, I guess I missed that one...bummer, dude:-[]


\section*{ 2016.06.15at21:41 }
\subsection*{ artificialScience thatLastOneIsFunnyBut getKyleInScience then getKyleNScience }
I have an idea I want to pursue.
I suspect Roundy could be helpful in this.
I want to develop a way for humans to get more intimate with the beauty that computers can provide through:
\begin{enumerate}
\item interacting with a beautiful operation system;
\item creating a logical entity as a pet with that operating system that is everywhere and can do anything between, alongside and through its host(computer)'s inputs and outputs;

\section*{ 2016.06.15at21:54 }
\subsection*{ ideasForScientificInvestigation }
I want to investigate something I will describe starting with the particular example whose consideration led me to think of the object of the investigation I hope to pursue.
Let a hollow glass cylinder be filled with a given liquid.
Orient the cylinder so its axis is vertical and close the top of its hole.

This liquid has an interesting property that appears on consideration to be characteristic of the liquid's mass density per unit of force of interparticle attraction.

That property is the distance which a unit mass will fall under its own weight.

\section*{2016.05.03at21:13}
\subsection*{iWantToWritebABook}

\section*{ 2016.05.18at23:30 }
\subsection*{iNight(lyRoutine)}

\section*{ 2016.05.19at08:55 }
\subsection*{coffeeCommuteAllenBrosOnMonroe}
I would like to take the idea of putting linux on a chromebook to Google. The hardware supports me sitting in the sun and type 16 hours on a single charge. With a linux backend, this blows away anything currently sold as a laptop.
>>>>>>> 95734bd0fc6b6cfd3c06c34beb18370e9b491567

\section*{ 2016.06.21at20:17 }
\subsection*{ somethingGreatToThinkAboutLate[r] pleaseMeStartToReadTheseMessagesToThroughAndFromMe }
This is or was at one time a serious topic. I can't remember the topic now. Here is my note to myself about that now: I am being kind to myself instead of being the kind of guy who beats himself up over forgetting a great idea that could launch himself into financial independence.



\section*{ 2016.06.22at22:15 }
\subsection*{ talkToRoundy }
What if squeeky clean teeth echo at a frequency equal to an expression in $v_t$, the velocity of sound in teeth?

\section*{ 2016.07.04at21:40 }
\subsection*{ iWantTo lifePlan livinLifeOneDayAtATime }
I think these tag-words in the latex subsection's title of my journal entries is helpful to keep my mind focused on what I want to journal about. 

I want to talk to Roundy. He reminds me a lot of Drew in ways that I like, such as wanting more to seek the truth for a satement of his than think about the feelings of those around him.

\section*{ 2016.07.07at21:49 }
\subsection*{ millionDollarIdea highdeas }
Let consideration of the following use case paint a software to fill its own need.

I want to create a software to allow people to make local media-streaming services in their areas. Areas here means wifi-reach. It would be a miracle for this to happen on a scale where anyone could make content and publish it to the whole conscious population of the universe.\

I want to start walking on my weekends. I could make a trip up to Portland this weekend. This is weekend is 3 days long.

Every weekend is three days because I work a half compressed work-week. My billable week ends at noon on fridays. Every Monday through Thursday I work nine hours to total 36 hours. I split the eight hours I work on Friday to give four hours to this week to total 40 hours. The other four hours I charge to the next week, marking four in a text-field in the web-form timesheet ORAU has paid Oracle to make for them.



\section*{ 2016.07.07at22:12 }
\subsection*{ ohYeah aboutThatMillionDollarIdea }
I intend to make an app so that people can make portable little radio stations over wifi 802.11a-damn-near-z. With the capital generated by the previous venture, and also with the name-recognition from this venture, I will create a global stage with free admission, free display, and free discovery.

\section*{ 2016.07.07at23:18 }
\subsection*{ theSecretToImprovingComputing }
Don't build the language to fit the hardware. We have done that step. The next leap-frog maneuver in increasing efficiency of computing machines could be us using our current vantage point, seeing the high-level-tasks which are commonly required of a computer from programming in the current state, using this vantage point that shows the high-level-tasks and the frequency with wihich they are required for contemporary computation.

\section*{ 2016.07.08at10:59 }
\subsection*{ bread }
Let one part be one of the tin measuring cups I have. The cup is engraved 1/2 cup. However, I emptied it six times into a measuring cup and measured barely less than two cups. So, one part is effectively 1/3 cup.
\begin{center}
\begin{tabular}{|r|l|}
\hline
Recipe:&Bread\\
\hline
flour           &   8parts=2+2/3cups\\
ripe bananas    &   4\\
yeast           &   1part=1/3cup\\
raisins         &   2parts=2/3cup\\
flax seeds      &   sprinkled atop once risen\\
NaCl            &   sprindled atop once risen\\
\hline
\end{tabular}
\end{center}

\section*{ 2016.08.02at09:03 }
\subsection*{ millionDollarIdea }
Tell Facebook about Kore. Point out the beauty of being able to feed those watching my digital stream the good things I take from knowing Kore, like how to be anabashedly, giddily happy. Talk to Facebook about how they provide a wonderfully accessible platform through which people can purposefully socially evolve how they choose to via feeding and being fed digitally. Take their job-offer.

\section*{ 2016.08.02at09:08 }
\subsection*{ purposefulEvolution }
A mandolin-player I jammed with after jamming with a celtic music group in the first street river park in Corvallis two weeks ago told me he is studied biology in school. I asked him for one or two beautiful nuggets of knowledge that he found in his studies he said that every living thing is so intricate and beautiful that one could happily spend a lifetime getting lost in exploring that organism.

I have a related thought. What if I told you that a creature had a magnificent feature, so magnificent because it depends on a huge number of environmental factors beig just the way they are in order for the feature to function as beautifully optimally as it does? This seems far from uncommon because these environmental factors shaped the creature's evolution. What if I now told you that the creature chose to evolve this way?

\section*{ 2016.08.02at22:34 }
\subsection*{ iDontWannaLoseThatThought }
I don't wanna lose that feeling, more like. Man, I do love the computer. I need to write a check if other people like the computer, too, or if they knew enough to know they need to. 

I like to sing out into thee,
Oh, infinite black infinity.

My terminal's characters look so damn pretty! Also, I want to let an open-source project use github as the platform from which to transmit how to sing into infinity by recognizing that your data will be saved at least as long as anything any random person will end up doing. What if thought-energy can be stored up in writing in an analogous manner to how thermal energy from eating hot foods being absorbed by their eater had better have been investigated.

I need to pick a conversation with physics. There is one person I think could help me be much more productive than I would be able to be without her by the sheer excitement I gain by picturing her as perfect. I may just let her live the rest of my life, as is the expectation from each of our respective demographics, without ever knowing how often a beautiful thing that the imaginer imagines must exist has been thought of her. It is seen by observing that last comment that communicating in prose that isn't structured in an enexcusably grammatical-symbol-dense language, prose is a much less exact communication of meaning than a programming language.

\section*{ 2016.08.02at23:25 }
\subsection*{ noteToSelf makeA millionDollarIdea }
I need to create a way to paint sound with a linux machine as follows. The player is dancing with one of the most beautiful phenomena of all time: hysteresis. The performer plays with hysteresis.

How it works is this: the artist gets some input from the sound that has come before. The artist is making sound at the same time. The cool and more easily monetizable thing is this: the hardware people build can be sold conscionably. That is, sold without slowing what could be humans', or really life's, faster advancement by hindering the use of a technology.

One such thing to build is a usb to microphone pocket-sized device. Doesn't this device already sound like a smartphone? This build could be just hardware for sale because I want open-sourced development of the software.

During the time I wrote this, I have cooked an incredible meal. There was no beginning to the preparation of this meal in memory. I used napkins that my mother had for what feels like the duration of forever.

I built the raised-bed in my yard years ago for a blogging project in a sustainability course. I hauled several carloads of fill-dirt and horse manure in surplus household waste-baskets. It was interesting to consider how and why the car handled in the particular peculiar ways it did when it was loaded with fill dirt.

My mom gave me the jars I scooped the beans photographed in my recent post on Facebook.

\section*{ 2016.08.03at01:53 }
\subsection*{ letMeTellYouAboutThisMeal }
It's so good I have to begin taking in a knife and cutting-board tomorrow to create a chewable concoction. I describe tonight's invention.
\begin{itemized}
\item I began soaking the beans last night,
\item a blend of rice and lentils that soaking last night.
\item I sure hope I finish this later...
\end{itemized}

\section*{ 2016.08.04at22:59 }
\subsection*{ letsMotivatePeople haveGlobalInternet-BasedCollaborations hereCollaborationIsReferringToTheParticularInstanceJustLikeThereAreCompetitionsHeldToFindASingleVictor,CollaborationsCouldBeHeldToFurtherEveryoneswellbeing }
People would be motivated to become a hero for mankind, a superstar that makes sports-stars jealous.

\section*{ 2016.08.04at23:19 }
\subsection*{ iWantTo }
I want to dream the world beautiful enough that nothing else really matters beside how beautiful the world is. I want to look up how to 

\section*{ 2016.08.04at23:20 }
\subsection*{ iWantto }
I want to grow an herb garden in my room. I also want herb-gardens throughout the house. Let there be a plant in, on, and around every room of the house.

Reflecting solar energy upward, in a way that has a limiting case of a mirror on a roof, seems like a non-negligible transformation of the energy of solar light into heat, presumably. I think the difference lies in water's change in entropy as it changes in phase. This change of phase hides some energy called the latent heat of fission, or, in the case of condensation the analogous hidden energy is called the latent heat of fusion.

Let me explain a foundation topic before proceeding. When heat is added to to an ice-cube, the cube increases in temperature x degrees per hour, where x is how quickly energy is transformed into heat in the cube. A physical scientist who has a temperature unit named after him proved in a seminal experiment that the speed at which mechanical energy is done on something can be measured by its temperature.

Specifically, temperature increases with work done to a system. However, at the temperature of phase-transition, say where one might see ice turn into water, some of this heat seems aborbed because the thermometer shows the same temperature as the ice melts entirely, all the while dumping the same energy into heating the system in a given time. 

So basically I am casting my vote for you and all your friends to foster plants' growth all around you because they are the beatiful organisms that are evolving in a feedback loop with our little astronomical rock. Humans think they are the dominant organism because we do horrendous deeds like chopping and burning down in days what took millenia of a whole ecosystem's effort to produce. Photosynthetic organisms terraformed the earth so that all intelligent life that we know of could exist.

A dirth of plants seems to map so well to the problems of today. We've chopped down forrests for grazing land and seen the land become a desert, just as the state of California is currently being punished for supplying too much agricultural land too densely. There is too much of the carbon-monoxide-like gases in the atmosphere. What do plants breathe?

One clever check of an effect which could reasonably be expected to betray a shortage of plants actively maintaining the terraformed state of our Earth is to check meteorological data to see if there has been a drop in oxygen content while there has been a rise in greenhouse gases(GHGs).

I believe humans would do well to become plant stewards. I want to toil to water the soil so the lands that have my heart don't become one ocean-broken desert.

I really need to start coming back to these journal entries because now I think I found a thought that might benefit humanity.

\section*{ 2016.08.09at22:33 }
\subsection*{ Right now I am just checking the journal-status of this machine. }
Now I wonder 

\section*{ 2016.08.09at23:25 }
\subsection*{ high conversation with mom }
roof 250$/mo.
tax 335$/mo.
insurance 79
car insurance 75
water sewer 60
rubbish 19
cable and internet 189$/mo,
cell phones 305$/mo.
little bills like netflix and wow 50

\section*{ 2016.08.10at00:04 }
\subsection*{ dearDiary }
The idea I recently had for a musical instrument made by listening to and responding to previous journal-entries of any recordable information is my computer-science great idea of all time and likely my first gift to humanity: I will deliver for mankind softwares that I see it can benefit from. Working for the EPA has got to be one of my all-time perfectly ideal jobs: I work to protect American humans from their environment by creating software to fill the needs my clients ask me to at work, but I have asked myself "what if my portfolio was comprised of open-sourced-public-service-projects?"

The great computer science idea is what I call playing hysteresis like and instrument. 

I need to soon find a way to remember things on a regular basis. There are weekly, monthly, annual, and less freequent events I want to remember with greater certainty than I should have had. This is said from one entity waching a sequence of events transpire while communicating artistically with his fellows.

		I want to pursue the idea of inventing journaling in a cross-every-kind-of-computer-use-of-the-word-platform journaling as a way to have a conversation with oneself. It seens like the new users would begin having a conversation with their past selves. Later they might have more conversations with their future-selves.
>>>>>>> 84076905b12836bdb2d0a29c238b3b6d650f1e2b
